\clearpage
\phantomsection
\addcontentsline{toc}{chapter}{Abstract}
\chapter*{Abstract}
The restoration of monochromatic images, especially black-and-white photographs, is essential in fields such as historical preservation, digital media enhancement, and visual content creation. This project focuses on developing an automated system that adds realistic color to grayscale images using a combination of OpenCV techniques and deep learning algorithms. Leveraging the powerful image processing capabilities of OpenCV and the advanced learning abilities of convolutional neural networks (CNNs), we aim to create a robust framework for black-and-white image colorization.

Our methodology involves preprocessing images to enhance their quality, followed by training a deep learning model to learn the intricate patterns and details necessary for accurate color restoration. The performance of our model is evaluated using both quantitative metrics, such as Peak Signal-to-Noise Ratio (PSNR) and Structural Similarity Index (SSIM), and qualitative assessments, demonstrating its ability to restore colors while preserving the intrinsic qualities of the original images.

The results showcase significant improvements in the visual quality of the restored images, underscoring the effectiveness of integrating traditional image processing methods with modern deep learning techniques. This project advances the domains of image restoration and processing by demonstrating the potential of AI-driven methods to revitalize historical or monochromatic images, offering new possibilities for enhancing visual content across various domains.
\\

\textbf{\textit{Keywords}}: Monochromatic Image Restoration, Image Colorization, OpenCV, Deep Learning, Convolutional Neural Networks (CNNs), Image Processing, Historical Image Restoration, AI-Powered Image Enhancement, Visual Content Enhancement, Grayscale to Color Conversion, Computer Vision, Machine Learning, Peak Signal-to-Noise Ratio (PSNR), Structural Similarity Index (SSIM), Digital Media Restoration